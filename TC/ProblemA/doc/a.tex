% Created 2019-04-04 Thu 11:11
% Intended LaTeX compiler: pdflatex
\documentclass[11pt]{article}
\usepackage[utf8]{inputenc}
\usepackage[T1]{fontenc}
\usepackage{graphicx}
\usepackage{grffile}
\usepackage{longtable}
\usepackage{wrapfig}
\usepackage{rotating}
\usepackage[normalem]{ulem}
\usepackage{amsmath}
\usepackage{textcomp}
\usepackage{amssymb}
\usepackage{capt-of}
\usepackage{hyperref}
\usepackage{minted}
\author{João Luís Freire Fraga}
\date{\today}
\title{Problema A}
\hypersetup{
 pdfauthor={João Luís Freire Fraga},
 pdftitle={Problema A},
 pdfkeywords={},
 pdfsubject={},
 pdfcreator={Emacs 26.1 (Org mode 9.1.9)}, 
 pdflang={English}}
\begin{document}

\maketitle
\tableofcontents

\pagebreak{}
\section{Explicacao do problema}
\label{sec:org5688866}

O problema consiste em criar um programa que, tendo como input:

\begin{itemize}
\item Estados
\item Estados iniciais
\item Estados finais
\item Transicoes possiveis
\item Palavra
\end{itemize}

Diga se a palavra é reconhecida pelo automato.
Ser reconhecido pelo automato implica que, comecando num dos estados iniciais, consumindo todas as letras da palavra letra a letra da esquerda para a direita e efectuando as actulizacoes necessarias, existe pelo menos um estado que pertence ao conjunto dos estados finais.

\pagebreak{}
\section{Leitura de input}
\label{sec:org5dbaf10}
\subsection{Numero de estados}
\label{sec:orgb2cdc66}
\begin{minted}[]{ocaml}
  (* Number of states *)
  let (n:int) = Scanf.scanf " %d" (fun x -> x)
\end{minted}
\subsection{Cardinalidade estados iniciais}
\label{sec:orga1053d1}
\begin{minted}[]{ocaml}
  (* Cardinality of S *)
  let (cardS:int) = Scanf.scanf " %d" (fun x -> x)
\end{minted}
\subsection{Estados iniciais}
\label{sec:org168fb5a}
Para ler os estados iniciais usei uma lista de tuples de inteiros "(int * int)" (s, x):
\begin{itemize}
\item s: state \{1, \ldots{}, n\}
\item x: estado interno (inicialmente 0)
\end{itemize}

\begin{minted}[]{ocaml}
  (* Initial states *)
  let (is:(int * int) list) =
    let rec readInitialState lst (i:int) =
      if i = 0 then
        lst
      else
        let state = Scanf.scanf " %d" (fun x -> x) in
        let lst = lst@[(state, 0)] in
        readInitialState lst (i-1)
    in
    readInitialState [] cardS
\end{minted}
\subsection{Cardinalidade estados finais}
\label{sec:org64468e1}
\begin{minted}[]{ocaml}
  (* Cardinality of F*)
  let (cardF:int) = Scanf.scanf " %d" (fun x -> x)
\end{minted}
\pagebreak{}
\subsection{Estados finais}
\label{sec:org77d65f5}
Para ler os estados finais usei uma lista de inteiros
\begin{minted}[]{ocaml}
  (* Final states *)
  let (fs:int list) =
    let rec readFinalState lst (i:int) =
      if i = 0 then
        lst
      else
        let s = Scanf.scanf " %d" (fun x-> x) in
        let lst = lst@[s] in
        readFinalState lst (i-1)
    in
    readFinalState [] cardF
\end{minted}
\subsection{Numero de transicoes}
\label{sec:orgab8118c}
\begin{minted}[]{ocaml}
  (* Number of transitions *)
  let (m:int) = Scanf.scanf " %d" (fun x -> x)
\end{minted}
\pagebreak{}
\subsection{Transicoes}
\label{sec:orgd974cab}
Para ler as transicoes usei uma lista de tuples do tipo "(int * char * string * int * int * int)" (i, c, op, a, b, j):
\begin{itemize}
\item i: inicio
\item c: caracter consumido
\item op: operaçao de comparacao
\item a: valor a ser comparado
\item b: valor a ser actualizado
\item j: fim
\end{itemize}

\begin{minted}[]{ocaml}
  (* Transitions *)
  let (t:(int * char * string * int * int * int) list) =
    let rec readTransitions lst (k:int) =
      if k = 0 then
        lst
      else
        let i  = Scanf.scanf " %d" (fun x -> x) in
        let c  = Scanf.scanf " %c" (fun x -> x) in
        let op = Scanf.scanf " %s" (fun x -> x) in
        let a  = Scanf.scanf " %c" (fun x -> x) in
        let a = if a = '_' then (-1) else (int_of_char a - 48) in
        let b  = Scanf.scanf " %c" (fun x -> x) in
        let b = if b = '_' then (-1) else (int_of_char b - 48) in
        let j  = Scanf.scanf " %d" (fun x -> x) in
        let transition   = (i, c, op, a, b, j) in
        let lst = lst@[transition] in
        readTransitions lst (k-1)
    in
    readTransitions [] m
\end{minted}
\subsection{Palavra}
\label{sec:org1bc1ac7}
\begin{minted}[]{ocaml}
  (* Word *)
  let (w:string) = Scanf.scanf " %s" (fun x -> x)
\end{minted}
\section{Funcoes}
\label{sec:org392fd03}
\subsection{checkState}
\label{sec:org99cdc24}
Esta funcao percorre a lista de estados actuais e verifica se algum deles se encontra na lista de estados finais.
\begin{minted}[]{ocaml}
  (* This function checks if any of the states is a final state *)
  let rec checkState state finalState =
    match state with
    | [] -> false
    | (s, _)::tl ->
      if List.mem s finalState then
        true
      else
        checkState tl finalState
\end{minted}
\subsection{operation}
\label{sec:orge4717f0}
Esta função converte uma string operação no valor booleano do resultado dessa operacao usando x e a
\begin{minted}[]{ocaml}
  (* This function converts the variables and operation into a boolean expression *)
  let operation x op a =
    match op with
    | "_" -> true
    | "<" -> x < a
    | "<=" -> x <= a
    | "=" -> x = a
    | "!=" -> x <> a
    | ">=" -> x >= a
    | ">" -> x > a
    | _ -> false
\end{minted}
\pagebreak{}
\subsection{followE}
\label{sec:orgf0d5f4e}
Esta funcao adiciona todos os estados onde e possivel chegar a partir dos estados actuais usando uma transicao epsilon
\begin{minted}[]{ocaml}
  (* This function follows current state's e's *)
  let rec followE state newState transitions =
    match state with
    | [] -> newState
    | (s, x)::stl ->
      let newState =
        if not (List.mem (s, x) newState) then
          newState@[(s, x)]
        else
          newState
      in
      let rec followETransitions newState transitions =
        match transitions with
        | [] -> newState
        | (i, c, op, a, b, j)::ttl ->
          let newState =
            if c = '_' && s = i then
              if b = (-1) then
                if not (List.mem (j, x) newState) then
                  newState@[(j, x)]
                else
                  newState
              else
              if not (List.mem (j, b) newState) then
                newState@[(j, b)]
              else
                newState
            else
              newState
          in
          followETransitions newState ttl
      in
      let newState = followETransitions newState transitions in
      followE stl newState transitions
\end{minted}
\subsection{followAllE}
\label{sec:org3d28175}
Esta funcao executa a funcao anterior enquanto esta produzir estados novos
\begin{minted}[]{ocaml}
  (* This function follows all e's possible *)
  let rec followAllE state lastState transitions =
    let state = followE state [] transitions in
    if state <> lastState then
      let lastState = state in
      followAllE state lastState transitions
    else
      state
\end{minted}
\subsection{consumeL}
\label{sec:orgf77dce2}
Esta funcao segue todas as transicoes que partem dos estados actuais e consomem a letra l, quando permitido pela guarda, e actualiza os tornando todos estes estados resultantes, os estados actuais.
\begin{minted}[]{ocaml}
  (* This function follows the letters *)
  let rec consumeL l state newState transitions =
    match state with
    | [] -> newState
    | (s, x)::stl ->
      let rec followLTransitions newState transitions =
        match transitions with
        | [] -> newState
        | (i, c, op, a, b, j)::ttl ->
          let newState =
            if c = l && s = i && operation x op a then
              if b = (-1) then
                newState@[(j, x)]
              else
                newState@[(j, b)]
            else newState
          in
          followLTransitions newState ttl
      in
      let newState = followLTransitions newState transitions in
      consumeL l stl newState transitions
\end{minted}
\subsection{solution}
\label{sec:org1ded6a5}
Esta funcao utiliza todas as funcoes anteriores para decidir se a palavra w e reconhecida pelo automato.
Se a palavra foi totalmente consumida e um dos estados actuais e um estado final então sim, caso contrario nao
\begin{minted}[]{ocaml}
  (* This function checks if the word is recognized *)
  let rec solution w state finalState transitions =
    if w = "" || state = [] then
      let state = followAllE state state transitions in
      checkState state finalState
    else
      let l = String.get w 0 in
      let state = followAllE state state transitions in
      let state = consumeL l state [] transitions in
      let w = (String.sub w 1 ((String.length w) - 1)) in
      solution w state finalState transitions
\end{minted}
\section{Execucao}
\label{sec:org47991a5}
Imprimir a solucao para o stdout
\begin{minted}[]{ocaml}
  (* Get answer *)
  let () =
    if solution w is fs t then
      Printf.printf "YES\n"
    else
      Printf.printf "NO\n"
\end{minted}
\end{document}
