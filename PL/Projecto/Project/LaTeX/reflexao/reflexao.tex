\section{Reflexão Crítica}
\label{sec3:imp-test}

\subsection{Introdução}
\label{sec3:subsec:intro}
Neste capítulo será feita uma reflexão critica sobre o desenvolvimento do projeto, elencando  os  objetivos  propostos  e  objetivos  alcançados,  principais  problemas encontrados e como foram superados.

\subsection{Problemas Encontrados}
\label{sec3:subsec:problems}

Durante a realização do projecto foram sentidas dificuldades na tomada de decisões e na interpretação da linguagem do enunciado.

A implementação de uma gramática sem conflitos foi um processo trabalhoso e levou ao aumento da complexidade da mesma.

A tipagem foi também um ponto desafiante principalmente devido a introdução do tipo intervalo e de tipos definidos pelos utilizadores.

\subsection{Análise Crítica}
\label{sec3:subsec:review}

Após realização deste projeto, considerando a proposta feita inicialmente, consideramos que o núcleo implementado do compilador ficou um pouco reduzido, no entanto consideramos que o interpretador atingiu todas as funcionalidades esperadas e até outras adicionais, tal como funções e strings. Os testes implementados poderiam também ter sido mais exaustivos e implementar programas mais genéricos.

\subsection{Conclusão}
\label{sec3:subsec:outro}
De um modo geral, os objetivos inicialmente propostos para este projeto foram alcançados.  A boa interação e organização entre os vários elementos do grupopermitiu o bom desenvolvimento das atividades definidas para este projeto.
