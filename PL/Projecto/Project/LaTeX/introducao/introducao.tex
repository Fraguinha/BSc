\section{Introdução}
\label{sec1:intro}

\subsection{Enquadramento}
\label{sec1:subsec:square}

Este projeto foi desenvolvido no âmbito da unidade curricular de Processamento de Linguagens, com o objetivo de construir um \textbf{interpretador} e um \textbf{compilador} para a linguagem \textbf{Natrix}.
A Linguagem Natrix é uma linguagem simples para computação numérica elementar, que trabalha com inteiros (64 bits) e intervalos de inteiros positivos. Tem vetores definidos com base nestes intervalos, dispõe também de uma estrutura condicional e de uma estrutura cíclica.
Finalmente esta linguagem Natrix tem funções (passagem por valor).

\subsection{Motivação}
\label{sec1:subsec:motive}

A realização deste projeto é motivada pelo desejo de aplicar os conhecimentos adquiridos ao longo do semestre na unidade curricular de Processamento de Linguagens e consolidar um pouco mais esses conhecimentos.

\subsection{Objetivos}
\label{sec1:subsec:goals}

Este projeto tem como objetivo desenvolver um interpretador e compilador para uma linguagem de programação elementar, \textbf{Natrix}, implementando de forma incremental:

\begin{itemize}
    \item Uma gramática (LR)
    \item Um \textit{lexer}
    \item Um \textit{parser}
    \item Uma semântica operacional
    \item Um interpretador
    \item Um compilador
\end{itemize}
