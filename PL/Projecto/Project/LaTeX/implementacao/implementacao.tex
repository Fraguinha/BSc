\section{Implementação e Testes}
\label{sec2:implementation}

\subsection{Introdução}
\label{sec2:subsec:intro}

Neste capítulo serão apresentadas detalhadamente as opções de implementação deste projeto, de forma a facilitar a sua interpretação, bem como alguns screenshots de algumas das principais funcionalidades implementadas.

\subsection{Gramática}
\label{sec2:subsec:choices}

Nesta secção serão apresentadas decisões tomadas para a linguagem.

\subsubsection{Comentários}

A linguagem Natrix permite o uso de comentários, tanto \textit{single line}, como \textit{multi line} através do uso dos símbolos "//", "(*" e "*)", respectivamente. Isto é representado pela expressão regular seguinte:

\lstinputlisting[firstline=7, lastline=7, caption={Comentarios}]{../../Grammar/natrix.ebnf}

Os comentários por bloco também ser aninhados o que não é possivel ser representado recorrendo apenas a esta expressão regular

\subsubsection{Variaveis}
\label{sec2:subsubsec:variaveis}

A nomenclatura de uma variavel é representada nas seguintes regras gramaticais:

\lstinputlisting[firstline=18, lastline=18, caption={identifier}]{../../Grammar/natrix.ebnf}

A declaração de variaveis é efectuada da seguinte forma:

\lstinputlisting[firstline=76, lastline=82, caption={declaração}]{../../Grammar/natrix.ebnf}

O valor de uma variavel pode ser alterado fazendo uso da regra de atribuição:

\lstinputlisting[firstline=84, lastline=87, caption={atribuição}]{../../Grammar/natrix.ebnf}

O acesso a uma variavel é feito recorrendo ao seu identificador ou, no caso dos arrays, um elemento pode ser acedido através do identificador e o indice correspondente

\lstinputlisting[firstline=20, lastline=20, caption={variavel}]{../../Grammar/natrix.ebnf}


\subsubsection{Tipos}

Na linguagem Natrix foram definidos varios tipos primitivos (null, int, bool, string) e foi permitida a definição de novos tipos através de combinações de tipos primitivos

\lstinputlisting[firstline=30, lastline=30, caption={tipos}]{../../Grammar/natrix.ebnf}

\subsubsection{Expressões}

Uma expressão é qualquer frase que represente um valor.

\lstinputlisting[firstline=58, lastline=74, caption={expressão}]{../../Grammar/natrix.ebnf}
\clearpage

\subsubsection{\textit{Statement}}

Um \textit{statement} é uma frase que representa uma instrução a executar.

Para além dos \textit{statements} já mencionados em variaveis~(\ref{sec2:subsubsec:variaveis} declarações e atribuições), estão também incluidos nos \textit{statements} a declaração de funções e instruções de controlo de fluxo (foreach, if then else)

\lstinputlisting[firstline=89, lastline=106, caption={statement}]{../../Grammar/natrix.ebnf}

\subsubsection{Programa}

Um programa na linguagem Natrix é representado por um conjunto de \textit{statements}

\lstinputlisting[firstline=108, lastline=108, caption={programa}]{../../Grammar/natrix.ebnf}
\clearpage

\subsection{Código}
\label{sec2:subsec:code}

\subsubsection{Lexer.mll}
\label{sec2:subsubsec:lexer}

O \textit{lexer} é gerado através do uso da ferramenta ocamllex~\cite{ocamllex}, neste são definidas quais as expressões regulares que serão reconhecidas pelo automato, nomeadamente:

\lstinputlisting[firstline=48, lastline=57, caption={Regular expressions}]{../../lexer.mll}
\clearpage

É também definido de que forma tratar cada uma das expressões regulares referidas anteriormente atraves da regra \textit{token}:

\lstinputlisting[firstline=59, lastline=101, caption={Regra token}]{../../lexer.mll}
\clearpage

De forma a obter um processamento mais eficiente no parser de \textit{identifiers}, é feito uso da função \textit{kwd\_or\_id} que implementa uma \textit{hashtable} de forma a verificar (e devolver) os \textit{tokens} correspondentes as palavras reservadas se estas forem reconhecidas, ou um \textit{token} genérico caso contrário:

\lstinputlisting[firstline=10, lastline=27, caption={id\_or\_kwd}]{../../lexer.mll}

Os \textit{ranges} são processados atrávez do uso da expressão regular \textit{range} e a função \textit{parse\_range} que divide a expressão regular anterior em 2 inteiros:

\lstinputlisting[firstline=29, lastline=42, caption={parse\_range}]{../../lexer.mll}
\clearpage

O processamento de \textit{strings} é feito através da regra \textit{string} que constroi uma string e devolve o token correspondente do seguinte modo:

\lstinputlisting[firstline=109, lastline=136, caption={Regra string}]{../../lexer.mll}

O \textit{lexer} também permite o reconhecimento de comentários aninhados através da regra \textit{comment} que faz uso de uma referência (\textit{level}) para manter o nível de profundidade do comentário:

\lstinputlisting[firstline=103, lastline=107, caption={Regra comment}]{../../lexer.mll}

\clearpage

\subsubsection{Parser.mly}
\label{sec2:subsubsec:parser}

O \textit{parser} é gerado através do uso da ferramenta menhir~\cite{menhir}, neste estão definidos os \textit{tokens} a ser utilizados, nomeadamente:

\lstinputlisting[firstline=7, lastline=19, caption={Tokens}]{../../parser.mly}

A prioridades e associatividade foram definidas como:

\lstinputlisting[firstline=21, lastline=35, caption={Prioridades e Associatividade}]{../../parser.mly}
\clearpage

As seguintes definições de regras são baseadas na gramática definida previamente~(\ref{sec2:subsec:choices}). Um programa é uma lista de \textit{statements} (ponto de entrada).

\lstinputlisting[firstline=37, lastline=47, caption={Regra program}]{../../parser.mly}

Os \textit{statements} foram separados nas regras \textit{statement} e \textit{simple\_statement}. No \textit{statement} são definidas as regras de construção de uma instrução \textit{if ... then ... else ...}, um ciclo \textit{foreach ... in ... do ...} e declaração de funções. Um \textit{statement} é também um \textit{simple\_statement} seguido de um ";".

\lstinputlisting[firstline=49, lastline=55, caption={Regra statement}]{../../parser.mly}

Na declaração de uma função, deve ser indicado o tipo de retorno bem como uma lista com todos os parametros e os tipos respectivos separados por ";". Ex:

\begin{lstlisting}
  function add : int (n1 : int; n2 : int) {
    return n1 + n2;
  }
  print(add(40 ; 2));
\end{lstlisting}

\lstinputlisting[firstline=57, lastline=58, caption={Regra parameters}]{../../parser.mly}

\lstinputlisting[firstline=60, lastline=61, caption={Regra single\_parameter}]{../../parser.mly}

Um \textit{block} é utilizado na construção do \textit{if} e \textit{for} para o agrupamos de 1 ou mais \textit{statement} dentro de chavetas.

\lstinputlisting[firstline=63, lastline=66, caption={Regra block}]{../../parser.mly}

Uma \textit{simple\_statement} é uma instrução que não é definida à custa de outras instruções.

\lstinputlisting[firstline=68, lastline=75, caption={Regra simple\_statement}]{../../parser.mly}

\lstinputlisting[firstline=77, lastline=80, caption={Regra set}]{../../parser.mly}
\clearpage

Uma expressão define uma instrução que representa um valor, tais como operações matemáticas, chamadas de funções e acesso a variáveis.

\lstinputlisting[firstline=82, lastline=93, caption={Regra expression}]{../../parser.mly}

Um \textit{array} é um tipo de conjunto de dados presente na linguagem Natrix definido recorrendo à seguinte regra:

\lstinputlisting[firstline=95, lastline=99, caption={Regra array}]{../../parser.mly}

Por fim, um \textit{type\_value} representa todas as construções de tipos.

\lstinputlisting[firstline=101, lastline=109, caption={Regra type\_value}]{../../parser.mly}

\clearpage

\subsubsection{AST.mli}
\label{sec2:subsubsec:ast}

A \textit{Abstract Syntax Tree} é definida atraves dos seguintes tipos:

\lstinputlisting[firstline=3, lastline=46, caption={AST}]{../../ast.mli}
\clearpage

\lstinputlisting[firstline=48, lastline=61, caption={AST}]{../../ast.mli}

Esta arvore representa a estructura de todos os programas possíveis da linguagem Natrix. Uma determinada instanciação desta permite a interpretação e a compilação de frases da linguagem.

\clearpage

\subsubsection{Main.ml}
\label{sec2:subsubsec:main}

O ficheiro main.ml é o ponto de entrada do programa que trata de juntar todas as peças do interpretador e compilador

\lstinputlisting[firstline=37, lastline=65, caption={Main.ml}]{../../main.ml}

O ficheiro de entrada é definido da seguinte forma:

\lstinputlisting[firstline=19, lastline=29, caption={file}]{../../main.ml}
\clearpage

Os erros são apresentados fazendo uso da seguinte função:

\lstinputlisting[firstline=31, lastline=35, caption={report}]{../../main.ml}

\clearpage

\subsubsection{Interpret.ml}
\label{sec2:subsubsec:interpret}

O ficheiro interpret.ml é o responsável por interpretar a árvore de síntaxe abstracta e executar código equivalente na linguagem OCaml.

\vspace{0.5cm}

\textbf{Núcleo implementado}

\vspace{0.2cm}

Foi implementado um interpretador capaz de lidar com todas as funcionalidades definidas na gramática estabelecida~(\ref{sec2:subsec:choices})

\vspace{0.5cm}

\textbf{Detalhes de implementação}

O armazenamento de varíaveis foi feito através de uma \textit{hashtable} do tipo (identifier, (value * typ)) que atribui a um identificador o seu valor e tipo.

O armazenamento de varíaveis de tipo foi feito através de um \textit{hashtable} do tipo (identifier, typ) que atribui a um identificador o tipo respectivo.

A declaração de funções foi feita, uma vez mais, através do uso de uma \textit{hashtable} do tipo (identifier, (typ * (identifier * typ) list * statement)) que atribui a um identificador o tipo de retorno da função, uma lista de parametros e os respectivos tipos e o código que esta contém.

Foram implementadas três funções principais para intrepretação do programa:

\begin{enumerate}
  \item \textbf{file} Itera sobre os \textit{statements} do programa
  \item \textbf{stmt} Recebe um \textit{statement} e realiza a sua execução
  \item \textbf{expr} Recebe uma expressão e retorna o seu valor
\end{enumerate}

Para toda esta implementação foi necessário a definição do tipo value que define os valores aceites pela linguagem Natrix.

\lstinputlisting[firstline=7, lastline=14, caption={Value}]{../../interpret.ml}


\subsubsection{Compile.ml}
\label{sec2:subsubsec:compile}

O ficheiro compile é o responsável por converter a árvore de síntaxe abstracta do ficheiro lido em código \textit{assembly}

\vspace{0.5cm}

\textbf{Núcleo implementado}

\vspace{0.2cm}

Foi implementado um compilador para uma versão simplificada da linguagem Natrix que implementa a função de Print, a atribuição de valores a variaveis e o acesso às mesmas, implementa também operações aritméticas sobre inteiros e \textit{bitwise operations}.

\clearpage

\subsection{Testes}
\label{sec2:subsec:testes}

Durante a implementação do intrepretador e compilador, foram criados vários testes (tanto positivos como negativos) de forma a ser possivel verificar que estes se estavam a comportar da forma esperada.

Os testes encontram se todos na pasta \textit{tests} numa de duas subdirectories:

\begin{enumerate}
  \item \textbf{good} (testes positivos)
  \item \textbf{bad} (testes negativos)
\end{enumerate}

De forma a tornar mais simples a verificação dos testes foi utilizado o script: \textit{run-tests}

\subsection{Manual de Utilização}
\label{sec2:subsec:manual}

O Interpretador e Compilador foi desenvolvido de forma a correr em qualquer plataforma onde seja possível correr \textit{OCaml}. O compilador gera codigo assembly para processadores \textit{X86\_64} e é executável no Sistema Operativo \textit{Linux}.

\subsubsection{Dependências}
\label{sec2:subsubsec:dependencias}

De forma a ser possivel compilar o projecto é necessário ter o programa \textit{Menhir} instalado, para tal, deverá ser executado o commando: \textit{\$ opam install menhir}

\subsubsection{Compilação}
\label{sec2:subsubsec:compilacao}

Para proceder a compilação do natrix deverá ser executado na linha de comandos: \textit{\$ make}

\subsubsection{Modos de utilização}
\label{sec2:subsubsec:utilizacao}

O programa Natrix possui dois modos de utilização:

\begin{enumerate}
  \item \textbf{interpretador} (\textit{\$ natrix --interpret filename.nx})
  \item \textbf{Compilador} (\textit{\$ natrix [-o file] filename.nx})
\end{enumerate}

O utilizador poderá também passar o comando: \textit{\$ ./natrix} para obter o menu de ajuda e visualizar todos os comandos descritos
